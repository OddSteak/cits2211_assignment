\documentclass[a4paper]{article}
\usepackage[margin=40pt]{geometry}
\usepackage{amsmath}

\begin{document}

\section*{Question 1}


\subsection*{a) P $\lor$ (Q $\lor$ $\neg$P)}

\begin{displaymath}
    \begin{array}{|c c| c | c | c |}
        \hline
        P & Q & \neg P & Q \lor \neg P & P \lor (Q \lor \neg P)\\
        \hline
        T & T & F & T & T\\
        T & F & F & F & T\\
        F & T & T & T & T\\
        F & F & T & T & T\\
        \hline
    \end{array}
\end{displaymath}

As the truth table demonstrates, the proposition is a tautology since
it's always true regardless of the truth values of Q or P


\subsection*{b) (P $\land$ $\neg$P) $\lor$ $\neg$Q}

P $\land$ $\neg$P is a contradiction since both P and it's negation can't be
true.\\
$\neg$Q can either be true of false\\
hence, the proposition is a contingent


\subsection*{c) Q $\implies$ (P $\land$ $\neg$Q)}

\begin{displaymath}
    \begin{array}{|c|c|c|c|c|}
        \hline
        P & Q & \neg Q & P \land \neg Q & Q \rightarrow (P \land \neg Q) \\
        \hline
        T & T & F & F & F \\
        T & F & T & T & T \\
        F & T & F & F & F \\
        F & F & T & F & T \\
        \hline
    \end{array}
\end{displaymath}

As the truth table demonstrates, the truth value of the proposition depends on
the truth values of P and Q.\\
Hence, the proposition is a contingent.


\section*{Question 2}
\subsection*{P $\lor$ $\neg$(P $\lor$ $\neg$Q) $\equiv$ P $\lor$ $\neg$Q}
By De Morgan's laws: P $\lor$ $\neg$(P $\lor$ $\neg$Q) $\equiv$ P $\lor$ ($\neg$P $\land$ Q)\\
By distributivity:\;\;\;\;\;\;\; P $\lor$ ($\neg$P $\land$ Q) $\equiv$ (P $\lor$ $\neg$P) $\land$ (P $\lor$ Q)\\
By excluded middle:\;\; (P $\lor$ $\neg$P) $\land$ (P $\lor$ Q) $\equiv$ T $\land$ (P $\lor$ Q)\\
By Absorption:\;\;\;\;\;\;\;\;\;\; T $\land$ (P $\lor$ Q) $\equiv$ P $\lor$ Q

\vspace{10pt}

\emph{Q.E.D.}

\section*{Question 3}
$\forall$x $\exists$n (x $\le$ n $\leq$ x+5 $\land$
($\exists$a())
)



\end{document}
