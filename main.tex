\documentclass[11pt]{article}
% \usepackage[margin=40pt]{geometry}
\usepackage[a4paper, margin=1in]{geometry}
\usepackage{amsmath}
\usepackage{lmodern}
\usepackage{bm}
% \usepackage{mathpazo}
\usepackage{fancyhdr}
\usepackage{lastpage}
\usepackage{amsfonts}
\usepackage{amssymb}

% \pagestyle{empty}
\title{\textbf{CITS2211 Assignment 1}}
\author{Name: Baasil Siddiqui \\ Student Id: 23895849}
\date{}

\pagestyle{fancy}
\fancyhf{}

\fancyfoot[R]{Page \thepage\ of \pageref{LastPage}}

\begin{document}
\parskip 2mm
\maketitle
\thispagestyle{fancy}

\section*{Question 1}
A \textbf{Tautology} is a compound proposition which is true under all possible
assignments of truth values of it's identifiers\\
\\
A \textbf{contradiction} is a compound proposition which is
false under all possible assignments of truth values of its identifiers\\
\\
A \textbf{contingent} proposition is one which is neither a tautology
nor a contradiction.
%
\subsection*{a) P $\lor$ (Q $\lor$ $\neg$P)}
The statement is a tautology because if P is true then so is $P \lor (Q \lor \neg P)$ \\
if P is false then, $Q \lor \neg P$ is true. Therefore, $P \lor (Q \lor \neg P)$ is true\\\\
The proposition is a tautology because it's true under all
possible assignments of the truth values of P or Q

\subsection*{b) (P $\land$ $\neg$P) $\lor$ $\neg$Q}

$(P \land \neg P) \lor \neg Q \equiv F \lor \neg Q$ (by contradiction) \\
$F \lor \neg Q \equiv \neg Q$ (by absorption) \\
\newline
The proposition is a contingent because it depends on the truth value of Q


\subsection*{c) Q $\rightarrow$ (P $\land$ $\neg$Q)}
if Q is true then, $(P \land \neg Q)$ is false, making the implication false, \\
and if Q is false the implication is true \\
\\
The proposition is a contingent because it depends on the truth value of Q

\section*{Question 2}
\subsection*{P $\lor$ $\neg$(P $\lor$ $\neg$Q) $\equiv$ P $\lor$ Q}

\setlength{\tabcolsep}{1em} % for the horizontal padding
{\renewcommand{\arraystretch}{1.5}% for the vertical padding
\begin{displaymath}
    \begin{tabular}{| l | l |}
        \hline
        1. P $\lor$ $\neg$ (P $\lor$ $\neg$ Q) & Premise \\
        \hline
        2. P $\lor$ ($\neg$ P $\land$ Q) & 1, Demorgan's laws \\
        \hline
        3. (P $\lor$ $\neg$P) $\land$ (P $\lor$ Q) & 2, distributivity \\
        \hline
        4. T $\land$ (P $\lor$ Q) & 3, excluded middle \\
        \hline
        5. P $\lor$ Q & 4, absorption \\
        \hline
    \end{tabular}
\end{displaymath}
\indent
5. P $\lor$ Q \\
\indent
\emph{Q.E.D.}

\section*{Question 3}
$\forall$n $\exists$x (n $\le$ x $\leq$ n+5 $\land$
($\exists$a $\exists$b (a $\neq$ x) $\land$ (b $\neq$ x)
$\land$ (a $\times$ b = x)))

\section*{Question 4}
let N(x, y) be the predicate "x is a neighbour of y"

\noindent
\textbf{a) Anna has no neighbours}
\par\parindent 20pt
$\neg$($\exists$\,x.N(x, a))

\noindent
\textbf{b) Ben has two neighbours}
\par\parindent 20pt
$\exists$\,x\,$\exists$\,y\,(N(x, b) $\land$ N(y, b) $\land$ x $\neq$ y $\land$
$\forall$\,z\,(N(z, b) $\rightarrow$ (z = x $\lor$ z = y)))

\noindent
\textbf{c) If somebody is a neighbour of Ben, Ben is also a neighbour of that person}
\par\parindent 20pt
$\forall$\,x\,(N(x, b) $\rightarrow$ N(b, x))

\noindent
\textbf{d) Except for Anna, everyone is the neighbour of someone}
\par\parindent 20pt
$\forall$\,x\,(x $\neq$ a $\rightarrow$ $\exists$y(N(x, y)))

\subsection*{Question 5}
An inference rule is sound iff assignment of truth values that
makes all the antecedents of the rule true must also make the
consequent true.\\

\noindent
\textbf{a)} \bm{$\displaystyle \frac{P}{P}$}

P $\equiv$ P by rule of identity\\
\indent
Therefore, the inference rule is sound
\parskip 6mm

\newpage
\noindent
\textbf{b)} \bm{$\displaystyle \frac{P}{P \leftrightarrow Q}$}

Counter example: if Q if false and P is true then, the axiom is true but the conclusion is false\\
\indent
Therefore, the inference rule is unsound

\noindent
\textbf{c)} \bm{$\displaystyle \frac{P \leftrightarrow Q}{P}$}

Counter example: if P and Q are false, the axiom is true but the conclusion is false\\
\indent
Therefore, the inference rule is unsound

\noindent
\textbf{d)} \bm{$\displaystyle \frac{P\;Q}{P \lor Q}$}

From the premise P we can infer P $\lor$ Q by disjunction introduction rule\\
\indent
Therefore, the inference rule is sound

\noindent
\textbf{e)} \bm{$\displaystyle \frac{P \rightarrow Q \; \neg \neg P}{Q}$}

\setlength{\tabcolsep}{1em} % for the horizontal padding
{\renewcommand{\arraystretch}{1.5}% for the vertical padding
\begin{displaymath}
    \begin{tabular}{| l | l |}
        \hline
        1. $\neg$$\neg$P & Premise \\
        \hline
        2. P $\rightarrow$ Q & Premise \\
        \hline
        3. P & 1, double negation \\
        \hline
        4. Q & 2, 3, Modus ponens \\
        \hline
    \end{tabular}
\end{displaymath}

Therefore, the inference rule is sound

\subsection*{Question 6}
$\forall$x.($\neg$ Q(x) $\lor$ P(x)) $\lor$
$\exists$ x(Q(x) $\lor$ (P(x) $\land$ R(x)))
$\rightarrow$ $\exists$x.R(x)
\setlength{\tabcolsep}{1em} % horizontal padding
{\renewcommand{\arraystretch}{2} % vertical padding
\begin{displaymath}
    \begin{tabular}{| l | l |}
        \hline
        1. $\forall x(\neg Q(x) \land P(x))$ & premise \\
        \hline
        2. $\exists x(Q(x) \lor (P(x) \land R(x)))$ & premise \\
        \hline
        3. $Q(a) \lor (P(a) \land R(a))$ & 2, Exist elimination \\
        \hline
        4. $\neg Q(a) \land P(a)$ & 1, Forall elimination\\
        \hline
        5. $\neg Q(a)$ & 4, conjunction elimination \\
        \hline
        6. $\neg Q(a) \to (P(a) \land R(a))$ & 3, implication law \\
        \hline
        7. $P(a) \land R(a)$ & 6, 5, Modus ponens\\
        \hline
        8. R(a) & 7, conjunction elimination \\
        \hline
        9. $\exists x.R(x)$ & 8, Exists introduction\\
        \hline
        Q.E.D. & \\
        \hline
    \end{tabular}
\end{displaymath}

\subsection*{Question 7}
Assume that the  difference beetween a squidgy and a non-squidgy number \\
multiplied by 2 produces a squidgy number

\noindent
let k be a non-squidgy number and l be a squidgy number \\
If l is a squidgy number there exists integers p and q such that \\
l = $\frac{p}{q}$

\noindent
{\begin{displaymath}
    (\frac{p}{q} - k) * 2 = \frac{p\prime}{q\prime}
\end{displaymath}
}
where $p\prime$ and $q\prime$ are integers (by our assumption that the result is a squidgy number)

\noindent
{\begin{displaymath}
    k = \frac{2pq\prime - p\prime q}{2 q\prime q} \text{ (By arithmetic)}
\end{displaymath}
}

\noindent
2p$q\prime$ - $p\prime$q and 2$q\prime$q are integers

\noindent
We showed that k can be represented in the form of $\frac{a}{b}$ where a and b are  integers, \\
therefore, we have a contradiction as k is a non-squidgy number yet, \\
can be represented as a fraction of integers \\
Our assumption that the result is a squidgy number must be false. \\
Therefore, the result of multiplying the difference betweeen a squidgy and a
non-squidgy number by 2 is a non-squidgy number \\
Q.E.D.

\subsection*{Question 8}
Definition: even numbers are divisible by 2 while odd numbers are not. \\ \\
this is a bidirection proof so firstly, we prove that if x is odd
then 5x-1 is even, \\
and then secondly, prove that if 5x-1 is even then x is odd. \\
\\
\textbf{first direction: if x is odd then 5x-1 is even} \\
\\
let x be an arbitary odd integer then x is of the form 2k-1
where k is an integer \\
\begin{align*}
    x &= 2k-1 \\
    5x &= 10k - 5 \text{  (multiplying both sides by 5)} \\
    5x - 1 &= 10k - 6 \text{  (substracting 1 from both sides)} \\
    &=2(5k - 3) \text{  (factoring out 2)} \\
    &=2 k \prime \text{  (where $k \prime$ = 5k-3)} \\
\end{align*}
5x-1 is of the form $2 k \prime$ which is an even number.\\
therefore, for any odd integer x, 5x -1 is an even integer

\noindent
\textbf{second direction: if 5x-1 is an even integer then x is an odd integer} \\
\\
The contrapositive of the statement is - \\
if x is not an odd integer then 5x-1 is not an even integer \\
We need to prove that
if x is an even integer then 5x-1 is an odd integer \\
\\
let x be an arbitary even integer then x is of the form 2k
where k is an integer \\
\begin{align*}
    x &= 2k \\
    5x &= 10k \text{  (multiplying both sides by 5)} \\
    5x - 1 &= 10k - 1 \text{  (substracting 1 from both sides)}  \\
    &= 2(5k)  - 1 \text{  (factoring out 2)} \\
    &= 2k\prime - 1 \text{  (where $k\prime$ = 5k)}
\end{align*}

\noindent
5x - 1 is of the form $2k\prime - 1$ which is an odd number \\
therefore, for any even integer x, 5x - 1 is an odd integer

\noindent
\textbf{Conclusion} \\
therefore, we have proved that if x is odd, then 5x-1 is even \\
and if 5x-1 is even then x is odd. \\
It follows that x is odd if and only if 5x-1 is even \\
Q.E.D.

\subsection*{Question 9}
let P(n) be "there exists an ordering of players $p_1, p_2...p_n$
such that $p_i$ defeats $p_{i+1}$ for all $i \in 1, 2, ... n-1$"\\\\
%
\underline{\textbf{Base case}}: \\
when n = 1, there is no valid value of i so the condition is satisfied for any arrangement \\
the 1 player can be arranged as "$p_1$" \\
Therefore, P(1) is true.
\\ \\
\underline{\textbf{Inductive case}}: \\
We want to show that $P(K) \rightarrow P(K+1)$ for some arbitary $k \geq 1$ \\
\\
\textbf{Inductive hypothesis}: \\
We can Assume that P(K) holds for an arbitary $K \geq 1$ \\
\\
\textbf{Inductive step:} \\
Now, we need to show that P(K+1) holds given the inductive hypothesis \\
\\
We can arrange the first K players as $p_1, p_2, \ldots, p_k$
such that $p_i$ defeats $p_{i+1}$ for all
$i = 1, 2, \ldots, k-1$. \\
note that we haven't considered any matches with $p_{k+1}$ yet \\
\\
Now, we need to add a new player $p_{k+1}$. Since $p_{k+1}$
plays a match against every other player, it either wins or loses each match.\\

\noindent
if $p_{k+1}$ beat all other players it can be inserted at the front.\\
The arrangement "$p_{k+1}, p_1, p_2 \ldots p_{k}$" will satisfy the condition \\
\\
Similarly, if $p_{k+1}$ loses to all other players it can be inserted at the end of the arrangement.\\
"$p_1, p_2 \ldots p_{k}, p_{k+1}$" will satisfy the condition\\
\\
Now, we will consider the final case, if $p_{k+1}$ defeated some players and lost to others.\\
Find a player $p_j$ among the existing $k$ players such that
$p_{j}$ defeats $p_{k+1}$ and $p_{k+1}$ defeats $p_{j+1}$ \\
\\
We know that such a value of j exists because the arrangement includes
players who beat $p_{k+1}$ and players who lost to $p_{k+1}$
so there must be a point where two such players are adjacent. \\
\\
Insert $p_{k+1}$ between $p_j$ and $p_{j+1}$, resulting in the new sequence:\\
$p_1, p_2, \ldots, p_j, p_{k+1}, p_{j+1}, \ldots, p_k$.\\
In this new order:\\
$p_i$ defeats $p_{i+1}$ for all $i = 1, 2, \ldots, j-2$.\\
$p_j$ defeats $p_{k+1}$.\\
$p_{k+1}$ defeats $p_{j+1}$.\\
$p_i$ defeats $p_{i+1}$ for all $i = j+1, j+2, \ldots, k$.\\
\\
Thus, we have arranged $k+1$ players such that $p_i$ defeats $p_{i+1}$
for all $i \in 1, 2, \ldots k$\\
Q.E.D.

\subsection*{Question 10}
\underline{\textbf{Base cases}}: \\
in the formula a,
A($\phi$) = 1 and B($\phi$) = 0 \\
in the formula baa,
A($\phi$) = 2 and B($\phi$) = 1

Thus, A($\phi$) $\geq$ 2B($\phi$) holds for the base cases

\noindent
\underline{\textbf{Inductive case $\psi$a$\phi$}}: \\
by the inductive Hypothesis, we can assume that
$A(\psi) \geq 2B(\psi) \text{ and } A(\phi) \geq 2B(\phi)$

\begin{align*}
    A(\psi a \phi) &= 1 + A(\psi) + A(\phi) \\
    &\geq 1 + 2B(\psi) + 2B(\phi) \\
    &= 1 + 2B(\psi \phi) \\
    &= 1 + 2B(\psi a \phi) \\
    &\geq 2B(\psi a \phi)
\end{align*}

\noindent
\underline{\textbf{Inductive case aba$\psi$}}: \\
by the inductive Hypothesis, we can assume that
$A(\psi) \geq 2B(\psi)$

\begin{align*}
    A(aba \psi) &= 2 + A(\psi) \\
    &\geq 2 + 2B(\psi) \\
    &= 2B(aba) + 2B(\psi) \text{ (since 2B(aba) = 2)} \\
    &= 2B(aba\psi)
\end{align*}
\\
Therefore, $\forall \phi . A(\phi) \geq 2B(\phi)$\\
Q.E.D
\subsection*{Question 11}
\textbf{Union}: The union of sets A and B is a set containing all elements appearing in either A or B or both.\\
\textbf{Intersection}: The intersection of sets A and B is a set containing all elements appearing in both A and B\\
\\
(i) $(A \cap \neg B \cap \neg C) \cup (B \cap C)$ \\ \\
(ii) $(A \cap \neg B \cap \neg C) \cup (B \cap \neg A \cap \neg C) \cup (C \cap \neg A \cap \neg B)$

\subsection*{Question 12}
(a) Definition: R is reflexive iff $\forall x \in X, R(x, x)$ \\ \\
%
I will disprove the given proposition with a counter-example: \\
let X = \{1, 2, 3, 4\} \\
Suppose R = \{(1, 2)(2, 3)(3, 4)(4, 4)\} and S = \{(2, 1)(3, 2)(4, 3)(4, 4)\} \\
then T = \{(1, 1)(2, 2)(3, 3)(4, 4)(3, 4)(4, 3)\} \\
Here, R and S are not reflexive since $1 \in X$ but (1, 1) $\notin$ R and (1, 1) $\notin$ S \\
but T is reflexive since $\forall x \in X, T(x, x)$

\noindent
(b) Definition: R is reflexive iff $\forall x \in X, R(x, x)$ \\ \\
%
For any arbitary element x in set X, \\
(x, x) $\in$ R and (x, x) $\in$ S (by defination of reflexive relations) \\
(x, x) belongs to T as well (by definition of T) \\
we have proved that for any arbitary element x in X, (x, x) $\in$ T \\
Therefore, if R and S are reflexive then so is T. \\
Q.E.D.

\noindent
(c) Definition: R is symmetric iff $\forall a, b \in X. ((a, b) \in R \rightarrow (b, a) \in R)$

\noindent
I will disprove the given proposition with a counter-example: \\
Suppose R = \{(1, 2)(2, 1)(3, 4)(4, 3)(1, 3)(3, 1)\} and S = \{(5, 6)(6, 5)(4, 3)(3, 4)\} \\
Then T = \{(3, 3)(4, 4)(1, 4)\} \\
here, R and S are symmetric \\
but T is not symmetric because (1, 4) $\in$ T but (4, 1) $\notin$ T \\

\end{document}
