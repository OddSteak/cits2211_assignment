\documentclass[11pt]{article}
\usepackage[margin=30pt]{geometry}
\usepackage{amsmath}
\usepackage{lmodern}
\usepackage{bm}
% \usepackage{mathpazo}
\usepackage{amsmath}
\usepackage{amsfonts}
\usepackage{amssymb}

\pagestyle{empty}
\title{\textbf{CITS2211 Assignment 1}}
\author{Baasil Siddiqui}
\date{}

\begin{document}
\parskip 2mm
\maketitle
\thispagestyle{empty}

\section*{Question 1}

\subsection*{a) P $\lor$ (Q $\lor$ $\neg$P)}

\begin{displaymath}
    \begin{array}{|c c| c | c | c |}
        \hline
        P & Q & \neg P & Q \lor \neg P & P \lor (Q \lor \neg P)\\
        \hline
        T & T & F & T & T\\
        T & F & F & F & T\\
        F & T & T & T & T\\
        F & F & T & T & T\\
        \hline
    \end{array}
\end{displaymath}
%
As the truth table demonstrates, the proposition is a tautology since
it's always true regardless of the truth values of Q or P


\subsection*{b) (P $\land$ $\neg$P) $\lor$ $\neg$Q}

P $\land$ $\neg$P = F (by contradiction)\\
$\neg$Q \hspace{36pt} (By absorption)\\
\newline
the proposition depends on the truth value of Q hence it's a contingent


\subsection*{c) Q $\rightarrow$ (P $\land$ $\neg$Q)}

\begin{displaymath}
    \begin{array}{|c|c|c|c|c|}
        \hline
        P & Q & \neg Q & P \land \neg Q & Q \rightarrow (P \land \neg Q) \\
        \hline
        T & T & F & F & F \\
        T & F & T & T & T \\
        F & T & F & F & F \\
        F & F & T & F & T \\
        \hline
    \end{array}
\end{displaymath}
%
As the truth table demonstrates, the truth value of the proposition depends on
the truth values of P and Q.\\
Hence, the proposition is a contingent.


\section*{Question 2}
\subsection*{P $\lor$ $\neg$(P $\lor$ $\neg$Q) $\equiv$ P $\lor$ $\neg$Q}

\setlength{\tabcolsep}{1em} % for the horizontal padding
{\renewcommand{\arraystretch}{1.5}% for the vertical padding
\begin{displaymath}
    \begin{tabular}{| l | l |}
        \hline
        1. P $\lor$ $\neg$ (P $\lor$ $\neg$ Q) & Premise \\
        \hline
        2. P $\lor$ ($\neg$ P $\land$ Q) & 1, Demorgan's laws \\
        \hline
        3. (P $\lor$ $\neg$P) $\land$ (P $\lor$ Q) & 2, distributivity \\
        \hline
        4. T $\land$ (P $\lor$ Q) & 3, excluded middle \\
        \hline
        5. P $\lor$ Q & 4, absorption \\
        \hline
    \end{tabular}
\end{displaymath}
\indent
5. P $\lor$ Q \\
\indent
\emph{Q.E.D.}

\section*{Question 3}
$\forall$x $\exists$n (x $\le$ n $\leq$ x+5 $\land$
($\exists$a $\exists$b (a $\neq$ n) $\land$ (a $\neq$ 1) $\land$ (b $\neq$ n) $\land$ (b $\neq$ 1)
$\land$ (a $\times$ b = n))

\section*{Question 4}
let N(x, y) be "x is a neighbour of y"

\noindent
\textbf{a) Anna has no neighbours}
\par\parindent 20pt
$\neg$($\exists$\,x.N(x, a))

\noindent
\textbf{b) Ben has two neighbours}
\par\parindent 20pt
$\exists$\,x\,$\exists$\,y\,(N(x, b) $\land$ N(y, b) $\land$ x $\neq$ y $\land$
$\forall$\,z\,(N(z, b) $\rightarrow$ (z = x $\lor$ z = y)))

\noindent
\textbf{c) If somebody is a neighbour of Ben, Ben is also a neighbour of that person}
\par\parindent 20pt
$\forall$\,x\,(N(x, b) $\rightarrow$ N(b, x))

\noindent
\textbf{d) Except for Anna, everyone is the neighbour of someone}
\par\parindent 20pt
$\forall$\,x\,(x $\neq$ a $\rightarrow$ $\exists$y(N(x, y)))

\subsection*{Question 5}

\noindent
\textbf{a)} \bm{$\displaystyle \frac{P}{P}$}

P $\equiv$ P by rule of identity\\
\indent
hence, the inference rule is sound
\parskip 6mm

\noindent
\textbf{b)} \bm{$\displaystyle \frac{P}{P \leftrightarrow Q}$}

if Q if false, the axiom is true but the conclusion is false\\
\indent
Hence, the inference rule is unsound

\noindent
\textbf{c)} \bm{$\displaystyle \frac{P \leftrightarrow Q}{P}$}

if P and Q are false, the axiom is true but the conclusion is false\\
\indent
Hence, the inference rule is unsound

\noindent
\textbf{d)} \bm{$\displaystyle \frac{P\;Q}{P \lor Q}$}

From the premise P we can infer P $\lor$ Q by disjunction introduction rule\\
\indent
Hence, the inference rule is sound

\pagebreak
\noindent
\textbf{e)} \bm{$\displaystyle \frac{P \rightarrow Q \; \neg \neg P}{Q}$}

\setlength{\tabcolsep}{1em} % for the horizontal padding
{\renewcommand{\arraystretch}{1.5}% for the vertical padding
\begin{displaymath}
    \begin{tabular}{| l | l |}
        \hline
        1. $\neg$$\neg$P & Premise \\
        \hline
        2. P $\rightarrow$ Q & Premise \\
        \hline
        3. P & 1, double negation \\
        \hline
        4. Q & 3, 2, Demorgan's laws\\
        \hline
    \end{tabular}
\end{displaymath}

Hence the rule is sound

\subsection*{Question 6}
$\forall$x.($\neg$ Q(x) $\lor$ P(x)) $\lor$
$\exists$ x(Q(x) $\lor$ (P(x) $\land$ R(x)))
$\rightarrow$ $\exists$x.R(x)
\setlength{\tabcolsep}{1em} % horizontal padding
{\renewcommand{\arraystretch}{3} % vertical padding
\begin{displaymath}
    \begin{tabular}{| l | l |}
        \hline
        1. $\forall x(\neg Q(x) \land P(x))$ & premise \\
        \hline
        2. $\exists x(Q(x) \lor (P(x) \land R(x)))$ & premise \\
        \hline
        3. $Q(a) \lor (P(a) \land R(a))$ & 2, Exist elimination \\
        \hline
        4. $\neg Q(a) \land P(a)$ & 1, Forall elimination\\
        \hline
        5. $\neg Q(a)$ & 4, conjunction elimination \\
        \hline
        6. $\neg Q(a) \to (P(a) \land R(a))$ & 3, implication law \\
        \hline
        7. $P(a) \land R(a)$ & 5, 6, Modus ponens\\
        \hline
        8. R(a) & 7, conjunction elimination \\
        \hline
        9. $\exists x.R(x)$ & 8, Exists introduction\\
        \hline
    \end{tabular}
\end{displaymath}

\subsection*{Question 7}
Assuming the difference beetween a squidgy and a non-squidgy number
multiplied by 2 produces a squidgy number

\noindent
we have a non-squidgy number k and a squidgy number l, \\
l can be represented as $\frac{p}{q}$ where p and q are integers

\noindent
{\begin{displaymath}
    (\frac{p}{q} - k) * 2 = \frac{p\prime}{q\prime}
\end{displaymath}
}
where $p\prime$ and $q\prime$ are integers (by the assumption that the result is a squidgy number)

\noindent
{\begin{displaymath}
    k = \frac{2pq\prime - p\prime q}{2 q\prime q}
\end{displaymath}
}
By arithmetic

\noindent
we know that the product of integers is an integer
the difference of integers is an integer as well

hence, 2p$q\prime$ - $p\prime$q is an integer and 2$q\prime$q is an integer as well

\noindent
hence k can be represented in the form of $\frac{a}{b}$ where a and b are  integers, \\
which contradicts with our premise that k is a non-squidgy number. \\
Our assumption that the result is a squidgy number must be false. \\
Hence, the result of multiplying the difference betweeen a squidgy and a
non-squidgy number by 2 is a non-squidgy number \\
Q.E.D.

\subsection*{Question 7}
Case 1: x is an odd integer \\
Case 2: x is an even integer

\end{document}

