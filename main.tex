\documentclass[a4paper]{article}
\usepackage[margin=40pt]{geometry}

\begin{document}

\section*{Question 1}
\subsection*{a) P $\lor$ (Q $\lor$ $\neg$P)}

\begin{displaymath}
    \begin{array}{|c c| c | c | c |}
        P & Q & \neg P & Q \lor \neg P & P \lor (Q \lor \neg P)\\
        \hline
        T & T & F & T & T\\
        T & F & F & F & T\\
        F & T & T & T & T\\
        F & F & T & T & T\\
    \end{array}
\end{displaymath}

as the truth table demonstrates the proposition is a tautology since
it's always true regardless of the truth values of Q or P

\section*{Question 2}
\subsection*{P $\lor$ $\neg$(P $\lor$ $\neg$Q) $\equiv$ P $\lor$ $\neg$Q}


\end{document}
